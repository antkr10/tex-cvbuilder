% Colors
\definecolor{pagebackground}{HTML}{FEFEFA}
\definecolor{sidebarbackground}{HTML}{ECEAEA}
\definecolor{sidebartitle}{HTML}{564E4E}
\definecolor{sidebartext}{HTML}{564E4E}
\definecolor{bodytitle}{HTML}{564E4E}
\definecolor{bodytext}{HTML}{564E4E}
\definecolor{bulletcolor}{HTML}{564E4E}
\definecolor{linkcolor}{HTML}{3E78B2}
\definecolor{linecolor}{HTML}{564E4E}
\definecolor{name}{HTML}{564E4E}
\definecolor{jobtitle}{HTML}{564E4E}
\RequirePackage[pagecolor=pagebackground, nopagecolor=pagebackground]{pagecolor}

% Heading styles
\titleformat{\section}
{\color{bodytitle}\fontsize{16pt}{16pt}\sffamily\addfontfeature{LetterSpace=2}}
{}{0pt}{}[{\color{linecolor}\titlerule[1pt]}]

\titleformat{\subsection}
{\color{sidebartitle}\fontsize{16pt}{16pt}\sffamily\addfontfeature{LetterSpace=2}}
{}{0pt}{}[{\color{linecolor}\titlerule[1pt]}]

\titleformat{\subsubsection}
{\color{bodytitle}\fontsize{12pt}{12pt}\sffamily\addfontfeature{LetterSpace=2}}
{}{0pt}{}

\titlespacing*{\section}{0pt}{0.5cm}{0.4cm}
\titlespacing*{\subsection}{0pt}{0.5cm}{0.4cm}
\titlespacing*{\subsubsection}{0pt}{0.5cm}{0.1cm}

% Defining font styles
\newcommand{\sname}{\color{name}\fontsize{36pt}{36pt}\sffamily\addfontfeature{LetterSpace=2}\MakeUppercase}
\newcommand{\sjobtitle}{\color{jobtitle}\fontsize{22pt}{22pt}\sffamily\addfontfeature{LetterSpace=5}\MakeUppercase}

% Defining page styles
\pagestyle{fancy}
\fancyhf{}
\renewcommand{\headrulewidth}{0pt}
\renewcommand{\footrulewidth}{0pt}
\fancyhead[C]{%
    \begin{tikzpicture}[remember picture,overlay]
        \node [rectangle, fill=sidebarbackground, anchor=north west, minimum width=8cm, minimum height=\paperheight+2cm] (box) at ($(current page.north west)-(1cm,-1cm)$) {};
        \draw [line width=1pt, color=linecolor] ($(current page.north west)+(7cm,2cm)$) -- ($(current page.south west)+(7cm,-2cm)$);
    \end{tikzpicture}
}

% Layout for name, job title and sidebar
\setcolumnwidth{5cm,12cm}
\setlength{\columnsep}{1.5cm}
\columncolor{sidebartext}[0]
\columncolor{bodytext}[1]

\newcommand{\makeprofile}{%
    \begin{paracol}{2}
        \switchcolumn[1]
            \ifthenelse{\equal{\profilepic}{}}{}{
                \begin{tikzpicture}[remember picture,overlay]
                    \node[rectangle, minimum height=4cm, minimum width=4cm, path picture={
                    \node at (path picture bounding box.center){\includegraphics[width=4cm]{\profilepic}};}] at ($(current page.north west)+(3.5cm,-2.5cm)$) {};
                \end{tikzpicture}
            }

            \hspace{-9.5cm}% Find a non-overlay way without ugly hspace
            \begin{tikzpicture}[remember picture,execute at begin node=\setstretch{2}]
                \node [rectangle, anchor=west, align=left, text width=12cm, minimum width=12cm, minimum height=2.5cm] (name) {\sname\name};
                \node [rectangle, line width=1pt, below=of name, yshift=.5cm, draw=bodytext, fill=pagebackground, align=left, text width=12cm, minimum width=\paperwidth+10cm, below, inner ysep=10pt] (title) {\sjobtitle\jobtitle};
            \end{tikzpicture}
    \end{paracol}
}

% Margins
\RequirePackage[left=1cm, right=1cm, top=1.5cm, bottom=1.5cm, footskip=0.5cm, headheight=0.5cm]{geometry}

% Personal skills bar
\newcommand\perskills[1]{
    \ifthenelse{\equal{#1}{}}{\booltrue{perempty}}{\renewcommand{\perskills}{%
        \foreach [count=\i] \x/\y in {#1}{
            {\x}\\
            \ifthenelse{\equal{\thecolumn}{0}}{
                \progressbar[width=.9\columnwidth,roundnessa=1pt, ticksheight=0, heightr=1, linecolor=linecolor, filledcolor=sidebartext, borderwidth=1pt, emptycolor=sidebarbackground]{\y}\par\smallskip
                }{
                \progressbar[width=.9\columnwidth,roundnessa=1pt, ticksheight=0, heightr=1, linecolor=bodytext, filledcolor=bodytext, borderwidth=1pt, emptycolor=pagebackground]{\y}\par\smallskip
            }
        }
    }}
}

% Professional skills bar
\newcommand\proskills[1]{
    \ifthenelse{\equal{#1}{}}{\booltrue{proempty}}{\renewcommand{\proskills}{%
        \foreach [count=\i] \x/\y in {#1}{
            {\x}\\
            \ifthenelse{\equal{\thecolumn}{0}}{
                \progressbar[width=.9\columnwidth,roundnessa=1pt, ticksheight=0, heightr=1, linecolor=linecolor, filledcolor=sidebartext, borderwidth=1pt, emptycolor=sidebarbackground]{\y}\par\smallskip
                }{
                \progressbar[width=.9\columnwidth,roundnessa=1pt, ticksheight=0, heightr=1, linecolor=bodytext, filledcolor=bodytext, borderwidth=1pt, emptycolor=pagebackground]{\y}\par\smallskip
            }
        }
    }}
}

% Command for drawing language skill circles
% Adapted from AltaCV Template: https://www.overleaf.com/latex/templates/altacv-template/trgqjpwnmtgv
\newcommand\langskills[1]{%
    \ifthenelse{\equal{#1}{}}{\booltrue{langempty}}{\renewcommand{\langskills}{%
        \renewcommand{\arraystretch}{1.4}
        \foreach [count=\i] \x/\y in {#1}{%
            \begin{tabular*}{\columnwidth}{@{}L{.4\columnwidth}L{.6\columnwidth}}
                {\x} &
                \foreach \z in {1,...,5}{%
                    {\ifnumgreater{\z}{\y}{\color{bodytext!50}}{\color{bodytext}}\small{\faCircle}}}\\
            \end{tabular*}
        }
    }}
}
